\documentclass[a4paper, 11pt]{article}

\usepackage[utf8]{inputenc}
\usepackage[slovene,english]{babel}
\usepackage[pdftex]{graphicx}
\usepackage{babelbib}


\newcommand{\todo}{\texttt{TODO}} % za TODOje pisat

\title{Pomnenje s kvantnimi celičnimi avtomati}
\author{Jasmina Pegan, Blaž Rojc}

\begin{document}
\maketitle

\section{Uvod}

Kvantne celice, osnovni gradniki kvantnih celičnih avtomatov (v nadaljevanju: QCA - \emph{Quantum-dot Cellular Automata}), same po sebi ne omogočajo pomnenja \cite{janez_phd}.
Podobno kot pri tranzistorjih mora načrtovalec (?) digitalnega vezja sestaviti strukturo iz kvantnih celic, ki pomnenje omogoči.
V tem delu bomo predstavili principe pomnenja v QCA in nekatere strukture iz kvantnih celic, ki simulirajo delovanje tradicionalnih pomnilnih celic.



\section{Pregled področja}

\subsection{Kvantni celični avtomati}

\todo: kaka citacija? šel sem po Janežu + Mrazu

Kvantni celični avtomati predstavljajo izvedbo celičnih avtomatov, v katerih so osnovni gradniki kvantne celice.
Kvantna celica je konstrukt kvadratne oblike, ki vsebuje štiri okrogle kvantne pike in štiri tunele.
Vsak tunel povezuje dve sosednji kvantni piki.
Shema takšne celice je prikazana na sliki \todo.

\todo: slika QC tukaj

V kvantni celici sta ujeta dva elektrona.
Vsak od njiju se lahko nahaja v eni izmed štirih kvantnih pik, med katerimi se lahko pomika prek tunelov, ki jih povezujejo.
Elektrona se zaradi odbojnih sil med njima v odsotnosti zunanjih vplivov postavita v eno izmed dveh stabilnih stanj, prikazanih na sliki \todo.

\todo: slika dveh stabilnih stanj

Stanje $P = -1$ interpretiramo kot logično vrednost $0$, stanje $P = 1$ pa kot logično vrednost $1$ \cite{lent_1993}.


\subsection{Problematika pomnenja}

\subsection{Reverzibilno procesiranje (?)}



\section{Pomnilne celice}

\subsection{Obstoječe implementacije}

\subsection{Reverzibilne implementacije (?)}

\subsection{Ideje in popravki (?)}



\section{Implementacija}

\todo: struktura poglavja se določi po testiranju/implementaciji

\subsection{Osnovni gradniki}

\subsection{Sheme}

\subsection{Analiza delovanja}



\bibliographystyle{babplain}
\bibliography{literatura} 

\end{document}
