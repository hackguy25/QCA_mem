\documentclass[a4paper, 12pt]{article}

\usepackage[utf8]{inputenc}
\usepackage[slovene,english]{babel}
\usepackage[pdftex]{graphicx}

\newcommand{\todo}{\texttt{TODO}} % za TODOje pisat

\title{Pomnenje s kvantnimi celičnimi avtomati}
\author{Jasmina Pegan, Blaž Rojc}

\begin{document}
\maketitle

\section{Uvod}

Kvantne celice, osnovni gradniki kvantnih celičnih avtomatov (v nadaljevanju: QCA - \emph{Quantum-dot Cellular Automata}), same po sebi ne omogočajo pomnenja. (\todo: citacija?)
Podobno kot pri tranzistorjih mora načrtovalec (?) digitalnega vezja sestaviti strukturo iz kvantnih celic, ki pomnenje omogoči.
V tem delu bomo predstavili principe pomnenja v QCA in nekatere strukture iz kvantnih celic, ki simulirajo delovanje tradicionalnih pomnilnih celic.



\section{Pregled področja}

\subsection{Kvantni celični avtomati}

\subsection{Problematika pomnenja}

\subsection{Reverzibilno procesiranje (?)}



\section{Pomnilne celice}

\subsection{Obstoječe implementacije}

\subsection{Reverzibilne implementacije (?)}

\subsection{Ideje in popravki (?)}



\section{Implementacija}

\todo: struktura poglavja se določi po testiranju/implementaciji

\subsection{Osnovni gradniki}

\subsection{Sheme}

\subsection{Analiza delovanja}



\end{document}
